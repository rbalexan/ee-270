% Some helpful commands for making EE269 handouts

% The document class options
\documentclass[11pt,oneside,notitlepage]{article}

% Packages
\usepackage{verbatim}
\usepackage{epsfig}
\usepackage{enumerate}
\usepackage{amsmath}
\usepackage{color}
\usepackage{ulem}
\usepackage{enumitem}
\usepackage{epstopdf}
\usepackage{amsmath}
\usepackage{algorithm}
\usepackage{algorithmic}
\usepackage{amssymb}
\usepackage{caption}
\usepackage{fullpage}
\usepackage{enumerate}
\usepackage{graphicx}
\usepackage{float}
\usepackage{hhline}
\usepackage{etoolbox}
\usepackage{subcaption}
\usepackage{multicol}
\usepackage{wrapfig}

% Preamble (style parameters, etc)
\pagestyle{empty}

\setlength{\paperwidth}{8.5in}
\setlength{\paperheight}{11in}
\setlength{\textwidth}{6.5in}
\setlength{\textheight}{9.5in}

\setlength{\topmargin}{0in}
\setlength{\evensidemargin}{0in}
\setlength{\oddsidemargin}{0in}

\setlength{\headheight}{0in}
\setlength{\headsep}{0in}
% more useful abbreviations
% -------------------------
\newcommand{\R}{\mathbb{R}}
\newcommand{\C}{\mathbb{C}}
\newcommand{\btheta}{\mbox{\boldmath $\theta$}}
\newcommand{\bgamma}{\mbox{\boldmath $\gamma$}}
\newcommand{\bbeta}{\mbox{\boldmath  $\beta$}}
\newcommand{\balpha}{\mbox{\boldmath $\alpha$}}
\newcommand{\bDelta}{\mbox{\boldmath $\Delta$}}
\newcommand{\bdelta}{\mbox{\boldmath $\delta$}}
\newcommand{\bPsi}{\mbox{\boldmath   $\Psi$}}
\newcommand{\bphi}{\mbox{\boldmath   $\phi$}}
\newcommand{\bpi}{\mbox{\boldmath    $\pi$}}
\newcommand{\btau}{\mbox{\boldmath   $\tau$}}
\newcommand{\blambda}{\mbox{\boldmath $\lambda$}}
\newcommand{\bTheta}{\mbox{\boldmath $\Theta$}}
\newcommand{\bone}{\mbox{\boldmath   $1$}}
\newcommand{\Loewner}[0]{\preceq}
\newcommand{\Hessmat}{{\cal H}}
\newcommand{\Bmat}{{\bf B}}
\newcommand{\Amat}{{\bf A}}
\newcommand{\bx}{{\bf x}}
\newcommand{\gradv}{{\bf g}}
\newcommand{\cG}{{\cal G}}
\newcommand{\cS}{{\cal S}}
\newcommand{\cT}{{\cal T}}
\newcommand{\trace}{\mbox{\rm trace}}
\newcommand{\tv}{\tilde{v}}
\newcommand{\gammaC}{\gamma_C}

\def\noprint#1{}
\def\swcomment#1{{\em [SW: #1]}}
\def\dmcomment#1{{\em [DM: #1]}}
\def\swresolved#1{}
\def\dmresolved#1{}
\def\sparsa{SpaRSA\ }
\def\bi{\begin{itemize}}
\def\ei{\end{itemize}}
\def\beq{\begin{equation}}
\def\eeq{\end{equation}}
\def\eqnok#1{(\ref{#1})}

\usepackage{listings}
\usepackage{color} %red, green, blue, yellow, cyan, magenta, black, white
\definecolor{mygreen}{RGB}{28,172,0} % color values Red, Green, Blue
\definecolor{mylilas}{RGB}{170,55,241}

\lstset{language=Matlab,%
    %basicstyle=\color{red},
    breaklines=true,%
    morekeywords={matlab2tikz},
    keywordstyle=\color{blue},%
    morekeywords=[2]{1}, keywordstyle=[2]{\color{black}},
    identifierstyle=\color{black},%
    stringstyle=\color{mylilas},
    commentstyle=\color{mygreen},%
    showstringspaces=false,%without this there will be a symbol in the places where there is a space
    numbers=left,%
    numberstyle={\tiny \color{black}},% size of the numbers
    numbersep=9pt, % this defines how far the numbers are from the text
    emph=[1]{for,end,break},emphstyle=[1]\color{red}, %some words to emphasise
    %emph=[2]{word1,word2}, emphstyle=[2]{style},    
}

\newcommand{\solution}[1]{
\iftoggle{solution}{
\vspace{1pc}   
\textcolor{blue}{\textbf{SOLUTION:}}\\ #1  
}{}
}

\definecolor{lightgray}{gray}{0.5}


\def\ie{{\em i.e.,~}}
\def\eg{{\em e.g.,~}}
\def\cf{{\em cf.,~}}
\def\ea{{\em et al.~}}
% Command to display the header
\newcommand{\HandoutHeader}[1]{
\noindent{\Large{\textbf{Large Scale Matrix Computation, Optimization and Learning (EE270)}} \hfill Handout {#1}} \rule[1.0mm]{6.5in}{1mm} {\Large Fall 2019}
\newline\newline }

\newcommand{\HomeworkHeader}[2]{
\noindent{\newline\newline\newline\large{\textbf{Large Scale Matrix Computation, Optimization and Learning (EE270)}} \hfill HW \#{#1}} \rule[1.0mm]{6.5in}{1mm}
\\
{\Large Winter 2021} \hfill {\large Due: \textbf{#2}}\newline\newline }

\newcommand{\HomeworkSolutionsHeader}[2]{
\noindent{\newline\newline\newline\Large{\textbf{Large Scale Matrix Computation, Optimization and Learning (EE270)}}  HW \#{#1} Solution} \rule[1.0mm]{6.5in}{1mm} {\Large Winter 2021}\newline\newline }

\newcommand{\NameSunet}{
{\bf Name \underline{\hspace{5cm}} }  \hspace{2cm}   {\bf SUNet ID \underline{\hspace{5cm}} }
\vspace{0cm}
}

%\newcommand{\HomeworkSolutionsHeader}[1]{
%{\Large{\bfseries Homework \#{#1} solutions}}
%}

\newcommand{\MidtermHeader}[1]{
\noindent{\Large{\bfseries Large Scale Matrix Computation, Optimization and Learning (EE270)}}\newline
\rule[1.0mm]{6.5in}{1mm}
{\Large (Winter 2020)} \newline\newline\newline
{\Large{\bfseries Midterm}} \hfill {\large Date: \textbf{#1}}\newline\newline\newline
}

\newcommand{\MidtermSolutionsHeader}[1]{
\newline
{\Large{\bfseries Midterm Solutions}} \hfill {\large Date: \textbf{#1}}\newline\newline\newline
}


\usepackage{hyperref}
\newtoggle{solution}
%\togglefalse{solution} %uncomment to hide solutions
\toggletrue{solution} %uncomment to show solutions
% \newcommand{\solution}[1]{\iftoggle{solution}{\vspace{1pc} \textbf{SOLUTION:} \\* #1}{}}
% \newcommand{\solutionspace}[2]{\iftoggle{solution}{\vspace{1pc} \textbf{SOLUTION:} \\* #2}{\vspace*{#1}}}


\newcommand{\hm}{{\hat\mu}}
\newcommand{\hS}{{\hat\Sigma}}

\def\alg{\mathcal{A}}
\def\R{\mathbb{R}}
\def\w{\mathbf{w}}
\def\x{\mathbf{x}}
\def\y{\mathbf{y}}
\def\z{\mathbf{z}}
\def\p{\mathbf{p}}
\def\EE{\mathbb{E}}
\def\PP{\mathbb{P}}

\begin{document}

\HomeworkHeader{3}{Friday, February 26, 11:59pm}

% \NameSunet

\begin{itemize}

\item After a genuine attempt to solve the homework problems by yourself, you are free to collaborate with your fellow students to find solutions to the homework problems. Regardless of whether you collaborate with other students, you are required to type up or write your own solutions. Copying homework solutions from another student or from existing solutions is a serious violation of the honor code. Please take advantage of  the professor's and TA's office hours. We are here to help you learn, and it never hurts to ask!
% \vspace{2mm}
\item {\bf  The assignments should be submitted via Gradescope including your code attached as pdf}
\end{itemize}


\begin{enumerate}
    \item \textbf{Logistic Regression (20 pts)}\\\\
    In this question, we will study logistic regression, which is a popular method in machine learning. We let $w \in \mathbb{R}^p$ be our decision variable. $w$ represents weights for the columns of the $n \times p$ data matrix $X$, where the $i^{th}$ row of $X$ is $x_i^T$. Let
    \begin{align*}
        \sigma(a) \triangleq \frac{1}{1+e^{-a}}=\frac{e^{a}}{1+e^{a}}
    \end{align*}
    be the sigmoid/logistic function. This function is non-convex. However $-\log(\sigma(a))$ is convex, which arises in maximum likelihood estimation as we describe next.
    
    We assume that we collect data $x_i$, and a binary response variable $y_i$, which is the label, where
    \begin{align}
        y_i= \begin{cases}+1 \text{ with probability } \sigma(w^Tx_i)\\
        -1 \text{ with probability } 1-\sigma(w^Tx_i)
        \end{cases}. \label{eq:logisticmodel}
    \end{align}
    We can write the above in the compact form $p(y_i|w,x_i)=\sigma(y_i w^T x_i)$ since $\sigma(a)=1-\sigma(-a)$. If we collect the observations $\{y_i\}_{i=1}^n$, then the probability of observing this outcome is $p(y|w,X)= \prod_{i=1}^n p(y_i|w,x_i)$ and so the negative log-likelihood, which we will use as our objective function, is
    \begin{align*}
        \ell(w) \triangleq -\log(p(y|w,X))= \sum_{i=1}^n \log(1+e^{-y_i w^T x_i}).
    \end{align*}
    Note that minimizing negative log-likelihood is same as maximizing the likelihood. This corresponds to maximum likelihood estimation of the parameter $w$. Once $w$ is identified, we can use \eqref{eq:logisticmodel} to infer the label of a test data point $x$.
\begin{enumerate}
    \item Derive the gradient $\nabla \ell(w)$.
    \item Derive the Hessian $\nabla^2 \ell(w)$.
    \item Is the cost function $\ell(w)$ convex?
\end{enumerate}
 \solution{
 }
 \item \textbf{Logistic Regression for Spam E-mail Classification (20 pts)}\\\\
 Download the email spam data set , which is available at \url{https://github.com/probml/pmtk3/tree/master/data/spamData} in Matlab and plain text format. This set consists of 4601 email messages, from which 57 features have been extracted. These are as follows
 \begin{itemize}
     \item  48 features, in $[0 \; 100]$, giving the percentage of words in a given message which match a given word on the list. The list contains words such as ``business'', ``free'', ``george'', etc.
     \item 6 features, in $[0 \; 100]$, giving the percentage of characters in the email that match a given character on the list. The characters are ; ( [ ! \$ \#
     \item Feature 55: The average length of an uninterrupted sequence of capital letters (max is 40.3, mean is 4.9)
     \item Feature 56: The length of the longest uninterrupted sequence of capital letters ( max is 45, mean is 52.6)
     \item Feature 57: The sum of the lengths of uninterrupted sequence of capital letters (max is 25.6, mean is 282.2)
     \end{itemize}
     Load the data set using the provided link. In the Matlab version, there is a training set of size 3065 and a test set of size 1536. In the plain text version, there is no predefined testing/training set, you can randomly shuffle the data to pick a training set of size 3065 and use the rest for testing (or you can use the Matlab data along with scipy.io.loadmat).
     
     There are different methods to pre-process the data, e.g., standardize the columns of $X$. For this problem, transform the features using $\log(x_{ij}+0.1)$. One could also add some regularization to the loss function which can help generalization error but this is not necessary. Also note that you need to transform the labels from $\{0, 1\}$ to $\{1, -1 \}$.
 \begin{enumerate}
     \item Run gradient descent with a fixed step-size. Plot the value of the cost function at each iteration and find a reasonable step-size for fast convergence
     \item Repeat the previous part using gradient descent with momentum.
     \item Implement gradient descent with Armijo line search. This procedure is as follows: Assume that we are at the point $x_k$ and have a search direction $p_k$ (for gradient descent $p_k=-\nabla f(x_k)$). Then, the Armijo line search procedure is: 
     \begin{itemize}
         \item Pick an initial step-size $t$
         \item Initialize the parameters $0<\rho<1$ and $0<c<1$ (typical values are $c=10^{-4}$ and $\rho=0.9$)
         \item While $f(x_k+t p_k)>f(x_k)+ct \nabla f(x_k)^T p_k$, do $t \leftarrow \rho t$
         \item Terminate the procedure if $f(x_k+t p_k)\leq f(x_k)+ct \nabla f(x_k)^T p_k$
     \end{itemize}
     The test statement in the while loop is the Armijo condition. If $p_k=-\nabla f(x_k)$, then the test is accepted when $f(x_k+t p_k) \leq f(x_k)-ct \|\nabla f(x_k)\|_2^2$. In general, the second term is negative as long as $p_k$ is a descent direction. One can prove this linesearch procedure will terminate.
     
     Find a good estimate for the initial step-size by trial and error. A simple idea is to use the final step-size from the previous step, but this can be unnecessarily small. You may want to do this, but increase the step-size by a factor of 2.

 \end{enumerate}
 \solution{
 }
\newpage
 
 \item \textbf{Newton's Method is Affine Invariant (20 pts)}\\\\
 In this question, we will prove the affine invariance of Newton's method. Let $f: \mathbb{R}^n \rightarrow \mathbb{R}$ be a convex function. Consider an affine transform $y \rightarrow Ay+b$, where $A \in \mathbb{R}^{n \times n}$ is invertible and $b \in \mathbb{R}^n$. Define the function $g: \mathbb{R}^n \rightarrow \mathbb{R}$ by $g(y)=f(Ay+b)$. Denote by $x^{(k)}$ the $k^{th}$ iterate of Newton's method performed on $f$. Denote $y^{(k)}$ the $k^{th}$ iterate of Newton's method performed on $g$.
\begin{enumerate}
    \item Show that if $x^{(k)}=A y^{(k)}+b$, then $x^{(k+1)}=A y^{(k+1)}+b$.
    \item Show that Newton's decrement does not depend on the coordinates, i.e., show that $\lambda(x^{(k)})=\lambda(y^{(k)})$, where $\lambda(x)=(\nabla f(x)^T \nabla^2f(x)^{-1} \nabla f(x))^{1/2}$.
\end{enumerate}
Together, this implies that Newton's method is affine invariant. As an important consequence, Newton's method cannot be improved by a change of coordinates, unlike gradient descent.
\solution{
}

\item \textbf{Newton's Method for Convex Optimization (20 pts)}
\begin{enumerate}
    \item Implement Newton's method for the logistic regression problem in Problem 1. Plot the value of the cost function at each iteration and find a reasonable step-size for fast convergence.
    \item Implement randomized Newton's method with uniform sampling sketch, i.e., sampling rows of $H^{1/2}$ uniformly at random where $H$ and $H^{1/2}$ denote Hessian and its square-root respectively. Plot the value of the cost function at each iteration and find a reasonable step-size and sketch-size for fast convergence.
\end{enumerate}
\solution{
 }


\item \textbf{Fast Johnson-Lindenstrauss Transform (FJLT) using Hadamard Matrices (20 pts)}
\begin{enumerate}
    \item Construct an $128 \times 1024$ FJLT matrix as follows
    \begin{enumerate}
        \item[] Set $m=128$ and $n=1024$
        \item[] Define $H_1:= \left[\begin{array}{cc} 1 & 1 \\ 1 & -1  \end{array}\right]$
        \item[] Construct $H_{10} \in \mathbb{R}^{1024,1024}$ recursively via $H_{k+1} = \left[\begin{array}{cc} H_k & H_k \\ H_k & -H_k  \end{array}\right] $
        \item[] Generate $D$ as an $n\times n$ diagonal matrix of uniformly random $\pm 1$ variables (Rademacher distribution) 
        \item[] Generate an $m \times n$ uniform sub-sampling matrix $P$ scaled with $\frac{\sqrt{n}}{\sqrt{m}}$ (uniform sampling sketch)
        \item[] Form the FJLT matrix $S=\frac{1}{\sqrt{n}}PHD$.
        \end{enumerate}
    \item Verify that $S^T S$ is a multiple of identity. Scale $S$ appropriately if needed to obtain $S^T S = I$.
    \item Generate a data matrix $A$ of size $1024\times 10$ using i.i.d. standard Gaussian variables. Plot the singular values of $A$ and singular values of $SA$.
    \item In part (c), $SA$ is a Johnson-Lindenstrauss embedding of $10$ vectors (1024 dimensional column vectors of $A$) to dimension $128$. Verify that the pairwise distances are approximately preserved, i.e., there exists an $\epsilon>0$ 
    \begin{align}
      (1-\epsilon)\|A_i-A_j\|_2^2 \le  \|S(A_i-A_j)\|_2^2\le (1+\epsilon) \|A_i-A_j\|_2^2\,\,\forall i,j\,,\label{eq:fjlt}
    \end{align}
    where $A_i$ is the i-th column of $A$. Find $\epsilon^*$, the smallest value of $\epsilon$ that satisfy \eqref{eq:fjlt} for a single realization of the random construction. Note that the matrices $D$ and $P$ are constructed randomly, while $H$ is deterministic. What is the minimum, maximum, and mean value of $\epsilon^*$ in 100 random realizations of the construction?  \\
    Hint: The JL embedding property in \eqref{eq:fjlt} specifies $2d^2$ linear inequalities of the form $\epsilon \ge c_{ij}$, hence the smallest $\epsilon$ is $\epsilon^*=\max_{ij} c_{ij}$.
    \item Generate a data matrix $A$ of size $1024\times 10$, and a vector $b$ of size $1024\times 1$ using i.i.d. standard Gaussian variables. Solve the least squares problem
    \begin{align*}
        x^* = \arg \min_x \|Ax-b\|_2^2\,.
    \end{align*}
    Apply the FJLT to $A$ and $b$ as $SA$ and $Sb$. Solve the sketched least squares problem
        \begin{align*}
             \tilde x := \arg\min_x \|SAx-Sb\|_2^2\,.
        \end{align*}
    Find the Euclidean distance between the solutions, i.e., $\|\tilde x - x^*\|_2$, between their predictions, i.e., $\|A\tilde x - Ax^*\|_2$  and the approximation ratio $(\| A\tilde x -b \|_2^2)/(\| A x^* -b \|_2^2) $ of the objective value.   
\end{enumerate}
\solution{
}
\end{enumerate}
\end{document}